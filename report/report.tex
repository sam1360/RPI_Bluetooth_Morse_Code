\documentclass[11pt]{article}

\usepackage{myreport}
\usepackage{csourcelst}

\begin{document}

% Fakesection Setup
% Cover page has specific expectations regarding formatting
% Prepare parskip values before starting

\newlength\origparskip
\newlength\origparindent

\setlength{\origparskip}{\parskip}
\setlength{\origparindent}{\parindent}

\setlength\parskip{\baselineskip}
\setlength\parindent{0 in}


% Fakesection Content

CMPE-240 Final Project Report

By submitting this report, we attest that its contents are wholly my individual writing about this exercise and that they reflect the submitted code.
We further acknowledge that the permitted collaboration for this exercise consists only of discussion of concepts with course staff and fellow students; however, other than code provided by the instructor for this exercise, all code was developed by us.

Author: Brian Alexander Mejorado and Samuel Mosher

Performed: December 5th, 2016

Submitted: December 5th, 2016

Lecture Section: 01

Professor: Alessandro Sarra

TA: Kevin Millar, Adrian Cruzat, Humza Syed


% Fakesection Teardown

\setlength\parskip{\origparskip}
\setlength\parindent{\origparindent}

\break


\section{Abstract}
\label{sec:abstract}


This project's goal was to develop a system to send messages to a Raspberry Pi~3 (RPi3) over Bluetooth and translate them to Morse~code.
The translated message is then transmitted over the RPi3's~GPIO pins to produce a Morse~code message conforming to international Morse~code standards.
In order to accomplish this, two separate programs were created: one program to send the message (capable of being compiled and run on any device with BlueZ installed), and the other to receive the message on the RPi3.
Functions used for each discrete step were developed in separate files for reusability.
A program was also created to test the functionality of each step.
In the end, the system was fully-functional, able to deliver messages through Bluetooth and translate them into GPIO signals.


\section{Design Methodology}
\label{sec:design_methodology}


% TODO Design Methodology


\subsection{Hardware Configuration}
\label{sub:hardware_configuration}


Since realistically implementing a Bluetooth-based system on the RPi3 requires OS support, a direct interface to the hardware was not possible.
Attempts to do so violate the OS's security and segmentation protocols.
As such, in order to access the GPIO~pins for the duration of the lab, a separate, third-party library was necessary.
After some searching, the \href{http://wiringpi.com/}{wiringPi} library was discovered, providing precisely the interface to the GPIO that was needed.


The wiringPi library uses an alternative mapping of GPIO~pins.
For the purposes of this project, wiringPi pin~0 was chosen, which corresponds to GPIO~17 or physical pin~11 per the documentation \href{http://wiringpi.com/pins/}{found here}.
This means that, in order to wire an LED through the program, GPIO~17 must be wired to (alongside GPIO~15, acting as a ground).


To allow the receiving RPi3 to simultaneously use a serial UART connection and a Bluetooth connection, the Bluetooth must be rewired to the RPi3's mini-UART.
The method to accomplish this can vary by OS, but in this particular project (which uses Raspbian Jessie) that is accomplished by adding the line \code{dtoverlay=pi-miniuart-bt} to the config.txt file at the root directory.
In doing so, the RPi3 can be booted --- and the program therefore run --- via the UART console.


With the receiving RPi3 utilizing the previously determined parameters (output on GPIO~17, ground on GPIO~15, and UART connections), the hardware layout of the RPi3 will be the same as in Figure~\ref{fig:receive-gpio}.


% TODO Get image to put here
\begin{figure}[ht]
    \centering
    \includegraphics[width=1\textwidth]{img/receive-gpio.png}
    \caption{The GPIO pin layout on the receiving RPi3.}
    \label{fig:receive-gpio}
\end{figure}


Note that the hardware of the sending device has not been discussed yet.
This is because only the receiving RPi3 has functionality tied directly to the Raspberry Pi hardware.
Theoretically, following complete development, the code to send messages \emph{to} the RPi3 can be compiled and run on any device that supports the BlueZ API.
Although, for the purposes of this lab, another RPi3 will be used as the sending device, the portability of the code means that the hardware configuration the code will run on is almost entirely unknowable beyond the presence of a few APIs.


\subsection{Morse Code \& Bluetooth Configuration}
\label{sub:morse_code_&_bluetooth_configuration}


% TODO MORSE CODE & Bluetooth Configuration
% The title of this section might change later


\subsection{Program Design}
\label{sub:program_design}


% TODO Program Design

% VERY BASIC OUTLINE
% Talk about higher-level execution; execution of bt-send and bt-receive on different devices
% High-level UML Diagram (what the program looks like at a higher level)
% Discuss bt-send
%     RFCOMM connection establish
%     Receive and package user data
% Discuss bt-receive
%     RFCOMM connection establish
%     State Diagram (what we are trying to accomplish)
%     Describe Lookup Table system (create table then convert characters of input into array of references to table)
%     Describe mapping bits to ticks ON and OFF per Morse code spec in \ref{sub:morse_code_&_bluetooth_configuration}; refer to State Diagram
%     Algorithm sketch for blink_code
%     Algorithm for blink_code


As has been stated in Subsection~\ref{sub:hardware_configuration}, the system operates not in a single RPi3, but in two different devices.
One of these devices specifies a MAC address field, then attempts to connect to the given MAC address and, if successful, sends packets of user input to the connected device.
The other device (the receiving RPi3) accepts a connection with a sending device, takes and decodes packets of user input, and turns it into blinkable Morse~code.
It is natural to split up both subsystems, then, and compile them separately.
This separation is reflected in the higher-level UML diagram for the system in Figure~\ref{fig:bt-morse-uml}, which details the soft and physical architecture of the system.


\begin{figure}[ht]
    \centering
    \includegraphics{img/bt-morse-uml.png}
    \caption{The high-level UML diagram for the project.}
    \label{fig:bt-morse-uml}
\end{figure}


Through Figure~\ref{fig:bt-morse-uml}, it is clear that although the systems are intertwined in execution, they \emph{are} separate programs.


% Hmm, maybe put this in its own section?
The sending component can be handled by a \code{bt-send.c}.


\begin{figure}[ht]
    \centering
    \includegraphics{img/bt-morse-state-diagram.png}
    \caption{A state diagram for the program. Note the resemblance to the Morse code spec from Subsection~\ref{sub:morse_code_&_bluetooth_configuration}.}
    \label{fig:bt-morse-state-diagram}
\end{figure}


\begin{algorithm}[ht]
    \begin{algorithmic}[l]
        \Procedure{blink\_signal\_from\_input}{char *buffer, Morse *codes}
        \State $n \gets$ \code{length(buffer)}
        \State $\code{converted} \gets$ \code{get\_codes\_from\_input( buffer, $n$, codes )}
        \For{$ i \gets 0 \text{ to } n$}
            \State $\text{count} \gets 0$
            \If{$ \code{converted[i].code} = 0 $}
                \State delay 4 ticks
                \State continue to next loop
            \EndIf
            \While{$ \code{converted[i].code} \brshift \text{count} \neq 1 $}
                \State $\text{count} \gets \text{count} + 1$
            \EndWhile
            \For{$ j \gets \text{count to } 0 $}
                \If{the jth bit of \code{converted[i].code} is 1}
                    \State blink on
                \Else
                    \State blink off
                \EndIf
                \State delay 1 tick
            \EndFor
            \State blink off
            \State delay 3 ticks
        \EndFor
        \EndProcedure
    \end{algorithmic}
    \caption{Pseudocode for converting the input to a signal through GPIO.}
    \label{alg:blink_signal_from_input}
\end{algorithm}


\section{Results and Analysis}
\label{sec:results_and_analysis}


% TODO Results & Analysis


\section{Conclusions}
\label{sec:conclusions}


% TODO Conclusions


\section{Source}
\label{sec:source}


% TODO Programatically acquire source code



\end{document}
