\documentclass[11pt]{article}

\usepackage{myreport}
\usepackage{csourcelst}

\begin{document}

% Fakesection Setup
% Cover page has specific expectations regarding formatting
% Prepare parskip values before starting

\newlength\origparskip
\newlength\origparindent

\setlength{\origparskip}{\parskip}
\setlength{\origparindent}{\parindent}

\setlength\parskip{\baselineskip}
\setlength\parindent{0 in}


% Fakesection Content

CMPE-240 Final Project Report

By submitting this report, we attest that its contents are wholly my individual writing about this exercise and that they reflect the submitted code.
We further acknowledge that the permitted collaboration for this exercise consists only of discussion of concepts with course staff and fellow students; however, other than code provided by the instructor for this exercise, all code was developed by us.

Author: Brian Alexander Mejorado and Samuel Mosher

Performed: December 5th, 2016

Submitted: December 5th, 2016

Lecture Section: 01

Professor: Alessandro Sarra

TA: Kevin Millar, Adrian Cruzat, Humza Syed


% Fakesection Teardown

\setlength\parskip{\origparskip}
\setlength\parindent{\origparindent}

\break


\section{Abstract}
\label{sec:abstract}


This project's goal was to develop a system to send messages to a Raspberry Pi~3 (RPi3) over Bluetooth via the mini uart and then translate them to Morse~code.
The translated message is then transmitted over the RPi3's~GPIO pins to produce a Morse~code message conforming to international Morse~code standards.
In order to accomplish this, two separate programs were created: one program to send the message (capable of being compiled and run on any device with BlueZ installed), and the other to receive the message on the RPi3.
Functions used for each discrete step were developed in separate files for reusability.
A program was also created to test the functionality of each step.
In the end, the system was fully-functional, able to deliver messages through Bluetooth and translate them into GPIO signals.


\section{Design Methodology}
\label{sec:design_methodology}


\subsection{Hardware Configuration}
\label{sub:hardware_configuration}


Since realistically implementing a Bluetooth-based system on the RPi3 requires OS support, a direct interface to the hardware was not possible.
Attempts to do so violate the OS's security and segmentation protocols.
As such, in order to access the GPIO~pins for the duration of the lab, a separate, third-party library was necessary.
% After some searching, the \href{http://wiringpi.com/}{wiringPi} library was discovered, providing precisely the interface to the GPIO that was needed.
After some searching, the wiringPi library was discovered, providing precisely the interface to the GPIO that was needed.


The wiringPi library uses an alternative mapping of GPIO~pins.
% For the purposes of this project, wiringPi pin~0 was chosen, which corresponds to GPIO~17 or physical pin~11 per the documentation \href{http://wiringpi.com/pins/}{found here}.
For the purposes of this project, wiringPi pin~0 was chosen, which corresponds to GPIO~17 or physical pin~11 per the documentation.
This means that, in order to wire an LED through the program, GPIO~17 must be wired to (alongside GPIO~15, acting as a ground).


To allow the receiving RPi3 to simultaneously use a serial UART connection and a Bluetooth connection, the Bluetooth must be rewired to the RPi3's mini-UART.
The method to accomplish this can vary by OS, but in this particular project (which uses Raspbian Jessie) that is accomplished by adding the line \code{dtoverlay=pi-miniuart-bt} to the config.txt file at the root directory.
In doing so, the RPi3 can be booted --- and the program therefore run --- via the UART console.


With the receiving RPi3 utilizing the previously determined parameters (output on GPIO~17, ground on GPIO~15, and UART connections), the hardware layout of the RPi3 will be the same as in Figure~\ref{fig:receive-gpio}.


\begin{figure}[ht]
    \centering
    \includegraphics[width=0.75\textwidth]{img/receive-gpio.jpg}
    \caption{The GPIO pin layout on the receiving RPi3.}
    \label{fig:receive-gpio}
\end{figure}


Note that the hardware of the sending device has not been discussed yet.
This is because only the receiving RPi3 has functionality tied directly to the Raspberry Pi hardware.
Theoretically, following complete development, the code to send messages \emph{to} the RPi3 can be compiled and run on any device that supports the BlueZ API.
Although, for the purposes of this lab, another RPi3 will be used as the sending device, the portability of the code means that the hardware configuration the code will run on is almost entirely unknowable beyond the presence of a few APIs.


\subsection{Morse Code \& Bluetooth Configuration}
\label{sub:morse_code_&_bluetooth_configuration}


Morse~code was developed by Samuel F.~B. Morse, Joseph Henry,~and Alfred Vail in 1836.
The first time Morse was used, it worked by making indents on a strip of paper when the machine received electric current.
Originally, Morse had intended to only support numbers through the code; each number would correspond to a word, which could be retrieved by dictionary.
However, shortly after that was introduced, Alfred Vail expanded this standard to include both special characters and letters.
This made Morse~code take off due to the fact that it was much easier to translate.
Vail used frequency analysis to assign dots and dashes to the letters.


The international Morse~code standard specifies the following regarding time lengths: a dot ``\mcdot'' is 1 time unit and a dash ``\mcdash'' is 3 time units.
In between each dot and dash is a 1 unit break, the space at the end of each letter is a break of 3 units, and the end of each word is indicated by a 7 unit break.
The reason for specifying the time length as units is due to the fact that the breaks do not have any specified length.
This means they can be relative to the speed at which the operator can send and receive messages.


The bluetooth connection is done using RFCOMM communication protocol.
RFCOMM ends up as a similar connection to that of a serial connection.
This made it easy to transmit and receive streams of data over the Raspberry Pi 3's mini UART.
The other reason for the selection of RFCOMM was the relative ease of use that the protocol provided, meaning that it didn't devour to much development time.
Within the receiving Raspberry Pi 3 bluetoothctl must turn discoverable on, so that the sending Pi can transmit data to it.
THe port that the pi recieves on is the first one of the current open connection.


% The title of this section might change later


\subsection{Program Design}
\label{sub:program_design}


% VERY BASIC OUTLINE
% Talk about higher-level execution; execution of bt-send and bt-receive on different devices
% High-level UML Diagram (what the program looks like at a higher level)
% Discuss bt-send
%     RFCOMM connection establish
%     Receive and package user data
% Discuss bt-receive
%     RFCOMM connection establish
%     Describe Lookup Table system (create table then convert characters of input into array of references to table)
%     State Diagram (what we are trying to accomplish)
%     Describe mapping bits to ticks ON and OFF per Morse code spec in \ref{sub:morse_code_&_bluetooth_configuration}; refer to State Diagram
%     Algorithm sketch for blink_code
%     Algorithm for blink_code


As has been stated in Subsection~\ref{sub:hardware_configuration}, the system operates not in a single RPi3, but in two different devices.
One of these devices specifies a Bluetooth address field, then attempts to connect to the given MAC address and, if successful, sends packets of user input to the connected device.
The other device (the receiving RPi3) accepts a connection with a sending device, takes and decodes packets of user input, and turns it into blinkable Morse~code.
It is natural to split up both subsystems, then, and compile them separately.
This separation is reflected in the higher-level diagram for the system in Figure~\ref{fig:bt-morse-uml}, which details the soft and physical architecture of the system.


\begin{figure}[ht]
    \centering
    \includegraphics[width=1\linewidth]{img/bt-morse-uml.png}
    \caption{The high-level architectural diagram for the project.}
    \label{fig:bt-morse-uml}
\end{figure}


Through Figure~\ref{fig:bt-morse-uml}, it is clear that although the systems are intertwined in execution, they \emph{are} separate programs.


% Hmm, maybe put this in its own section?
The sending component can be handled by a \code{bt-send.c} file.
In order to ensure that the user provides a Bluetooth address, it makes sense to enforce that the send program accepts that Bluetooth address as a command-line argument when it is called.
Since command-line argument validation is a fairly common procedure in C programming, the design of that particular component will not be documented here.


After the Bluetooth address is validated, the connection needs to be established.
Using the BlueZ framework, we can create an RFCOMM Bluetooth connection as described in Subsection~\ref{sub:morse_code_&_bluetooth_configuration}.
Based on the documentation for BlueZ, this can be done by creating an RFCOMM-specific struct for the socket address, allocating the socket, setting connection parameters (e.g.: the provided Bluetooth address), and then connecting with that struct.
Bluetooth works over pre-defined ``channels''; since this is a relatively simple application, creating the RFCOMM connection over channel 1 should be sufficient.


Once the sending device establishes the connection, it should be enough to loop for user input, send the input over Bluetooth, and evaluate the status of the connection for each iteration.


% bt-receive is considerably more dense
The receiving component can likewise be handled by a \code{bt-receive.c} file.
Because of the nature of this program, considerably more work must be done for receiving.
Like with the program sending the information, the receiving device must prepare itself for the RFCOMM connection.
The program should bind the local socket to the first available Bluetooth adapter and then accept one connection from a sending device before running.
Once this is done, the program should loop until it receives input and then output the results.


\code{bt-receive.c} is also where the wiringPi library, mentioned in Subsection~\ref{sub:hardware_configuration}, must be initialized.
As stated, wiringPi Pin~0 must be configured to output.


After wiringPi is configured, it is necessary to finally address how \code{bt-receive.c} will translate and encode input into Morse~code.
To do so, separate modules for conversion and blinking must be developed.


Morse~code is an international communication standard that has been around for 180~years as of the time of writing.
As such, there is little reason for the standard to change.
The invariability of Morse~code's design can be taken into consideration in program design.
Rather than manually convert each individual letter into Morse~code, it is much more efficient to pre-allocate a table of the already-established Morse~code values and index into that table.
For the conversion, a struct can be defined which contains a Morse~code string and the corresponding ASCII value.
Then, during the program initialization, a table of pointers to structs of this format can be created.
Using this, each character in the input can be mapped to one of the references in the table, allowing the input to be completely converted without issue.
This process of mapping is illustrated in Algorithm~\ref{alg:get_codes_from_input}, which describes the function responsible for the conversion.


\begin{algorithm}[ht]
    \begin{algorithmic}[l]
        \Procedure{get\_codes\_from\_input}{char *buffer, size\_t $n$, Morse *codes}
            \State $\code{converted} \gets \text{ allocated space for n codes}$
            \For{$ i \gets 0 \text{ to } n $}
                \State $\code{index} \gets \code{get\_codes\_index( buffer[i] )}$
                \State $\code{converted[i]} \gets \code{codes[index]}$
                \State print value of $\code{converted[i].convStr}$
            \EndFor
            \State print a newline
            \State \Return \code{converted}
        \EndProcedure
    \end{algorithmic}
    \caption{Pseudocode for mapping to the codes table.}
    \label{alg:get_codes_from_input}
\end{algorithm}


All characters in the input can be converted individually to Morse~code signals, which consists of dots and dashes (or ``dits'' and ``dahs'').
Refer back to the description of how Morse~code is timed in Subsection~\ref{sub:morse_code_&_bluetooth_configuration}.
Each dot is 1 time unit on and each dash is 3 time units on.
Between dots and dashes there is 1 time unit off, between letters there are 3 time units off, and between words there are 7 time units off.
This gives us 5 possible states for the signal to be in: 1 tick on, 3 ticks on, 1 tick off, 3 ticks off, and 7 ticks off.
If we represent the next portion of the Morse~code-translated input as a transition, we arrive at a State machine for translating input to Morse~code signals, as seen in Figure~\ref{fig:bt-morse-state-diagram}.


\begin{figure}[ht]
    \centering
    \includegraphics[width=0.8\linewidth]{img/bt-morse-state-diagram.png}
    \caption{A state diagram for conversion. Note similarities with the Morse code spec from Subsection~\ref{sub:morse_code_&_bluetooth_configuration}.}
    \label{fig:bt-morse-state-diagram}
\end{figure}


This state machine represents the effective results of the program as it goes from accepting input to translating into Morse to interpreting the translated output.
To interpret the translated output, an additional parameter can be added to the structs in the table.
Each tick needed to produce a symbol can be represented in a binary code which is stored in the struct.
For example, the letter ``S'' would provide the code ``\mcdot{} \mcdot{} \mcdot{}'' which becomes \bina{10101}, or \hex{0x15} in hex.
Then, an algorithm to blink the letter would traverse this binary code in bitwise fashion to blink on and off for the correct amount of time.
After printing the letter, the algorithm would then force the LED to blink off for 3 ticks, and an additional 4 ticks if it has reached the end of the word.


Including the conversion operation from Algorithm~\ref{alg:get_codes_from_input} and details for bitwise traversal, following this procedure would create the pseudocode shown in Algorithm~\ref{alg:blink_signal_from_input}.
Implementing that algorithm fully converts user input into Morse~code signals through the selected GPIO pin, thus accomplishing the goal of the system as a whole.


\begin{algorithm}[ht]
    \begin{algorithmic}[l]
        \Procedure{blink\_signal\_from\_input}{char *buffer, Morse *codes}
        \State $n \gets$ \code{length(buffer)}
        \State $\code{converted} \gets$ \code{get\_codes\_from\_input( buffer, $n$, codes )}
        \For{$ i \gets 0 \text{ to } n$}
            \State $\text{count} \gets 0$
            \If{$ \code{converted[i].code} = 0 $}
                \State delay 4 ticks
                \State continue to next loop
            \EndIf
            \While{$ \code{converted[i].code} \brshift \text{count} \neq 1 $}
                \State $\text{count} \gets \text{count} + 1$
            \EndWhile
            \For{$ j \gets \text{count to } 0 $}
                \If{the jth bit of \code{converted[i].code} is 1}
                    \State blink on
                \Else
                    \State blink off
                \EndIf
                \State delay 1 tick
            \EndFor
            \State blink off
            \State delay 3 ticks
        \EndFor
        \EndProcedure
    \end{algorithmic}
    \caption{Pseudocode for converting the input to a signal through GPIO.}
    \label{alg:blink_signal_from_input}
\end{algorithm}


\section{Results and Analysis}
\label{sec:results_and_analysis}


\begin{figure}[ht]
    \begin{minipage}[t]{0.49\linewidth}
        \centering
        \includegraphics[width=1\textwidth]{img/program-output-send.png}
        \caption{The program running on the sending device.}
        \label{fig:program-output-send}
    \end{minipage}
    \hspace{\fill}
    \begin{minipage}[t]{0.49\linewidth}
        \centering
        \includegraphics[width=1\textwidth]{img/program-output-receive.png}
        \caption{The program running on the receiving device.}
        \label{fig:program-output-receive}
    \end{minipage}
\end{figure}


\begin{figure}[ht]
    \centering
    \includegraphics[scale=0.5]{img/morse.png}
    \caption{A table displaying the conversion from ASCII to Morse~code.}
    \label{fig:international_morse_code}
\end{figure}


As displayed in Figures~\ref{fig:program-output-send} and~\ref{fig:program-output-receive}, the program ran as specified.
The program was required to take input from a user on and then proceed to transmit it over Bluetooth to the receiving Raspberry Pi.
From there the program needs to convert the ASCII text to dots and dashes, which is shown by the output.
The input and the received value was ``\code{hey whats up}''.
The program translated the code to ``\verb|**** * -*--   *-- **** *- - ***   *-- *--*|'' as shown in Figure~\ref{fig:program-output-receive}.
Using the International Morse~code table from Figure~\ref{fig:international_morse_code}, we find H=\mcdot{}\mcdot{}\mcdot{}\mcdot{}, E=\mcdot{} Y=\mcdash{}\mcdot{}\mcdash{}\mcdash{}, W=\mcdot{}\mcdash{}\mcdash{}, H=\mcdot{}\mcdot{}\mcdot{}\mcdot{}, A=\mcdot{}\mcdash{}, T=\mcdash{}, S=\mcdot{}\mcdot{}\mcdot{}, U=\mcdot{}\mcdot{}\mcdash{}, and P=\mcdot{}\mcdash{}\mcdash{}\mcdot{}.
This shows that the code is valid for converting characters to Morse.
The UART blink in operation is shown in Figure~\ref{fig:receive-gpio}, as well as the correct pinout to use when running the application.


\section{Conclusions}
\label{sec:conclusions}


The objective to connect two Raspberry Pi's together and have them communicate via Bluetooth was successful.
On top of that it was successful in converting from ASCII to Morse~code and then blinking it by using a GPIO pin to switch the output from low to high.
There are numerous applications for Bluetooth, especially with the rising popularity of Internet of Things devices.
With Bluetooth it allows people to connect more and more devices to either their phone or computer, and get data or control different parts of a system remotely.
This is an industry that is only just starting to take off, things like watches are becoming more and more connected due to the use of Bluetooth connectivity.
The main issues that were encountered during the creation of this project included getting Bluetooth working, the initial configuration of the Raspberry Pis, and conflicting schedules.
Bluetooth was solved by using resources online as well as just struggling through until the Pis connected to one another.
The configuration was solved in a similar way, mostly trial and web resources to get it running.
Regarding conflicting schedules, this was resolved by a really long weekend when we both finally had time to get together and work on the code.
If we were to do it over again, we would either try to find more time to meet, or pass off the Raspberry Pis to one another when one of us had free time to work on it.


\break
\section{Source}
\label{sec:source}


% TODO Programatically acquire source code



\end{document}
